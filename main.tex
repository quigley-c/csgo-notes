\documentclass{article}

\title{CS:GO Notes}
\date{}

\begin{document}
\maketitle

\section*{Preface}

	This document describes my basic understanding of how to play the game. This includes information
	such as what options are strongest in certain situations (eg. enemy AWP watching a long sightline).
	This was inspired by an idea I had to treat my teammates as RPG characters to 'deploy' them
	in situations where their strong points are advantaged. The second half of this document deescribes
	my observations of the strengths and weaknesses of each member of our team.

\section*{AIM FOR DUMMIES}

One thing I find generally helpful to keep in mind when aiming is to literally take my time with my shots. This has a couple
meanings, but most importantly it means that before I ever consider firing a bullet, I am going to confirm that my cursor is
all the way over to where I want it to be and that i'm not moving before I fire. Generally, the faster you can confirm that
your crosshair is correctly placed, the faster you can fire. So the whole motion takes the time of aiming and confirming your
target. Keep in mind that it feels like aiming takes a long time, and not to get hasty. It takes longer to aim an accurate shot
than it does to aim an inaccurate one. This means that you'll actually spend less time fighting overall however, since you will
be hitting more shots and killing your opponent more efficiently. When done quickly these look like flicks. Flicks are slightly
different in how they function, however, and I won't be covering them here yet as I don't think any description of flicking
I could provide right now would be beneficial.

\section*{Guidelines. Strategies, and Gameplay Techniques}

\subsection*{Trading}
Generally when attempting to make an aggressive play where clearing angles individually is not preferable or possible, players
will push in a team of 2 or more. The idea behind these pushes is to use your man advantage to increase the amount of time you
have to locate and kill an enemy. The reason it's so effective is because, when done properly, it near guarantees at least a
1 for 1 trade of players. Being able to force trades like this is important to keeping control of the pace of the match.

\subsection*{Holding Angles}
\begin{itemize}
    \item Generally, static holding angles is bad unless you've got a scope, improving your lethality at range due to better accuracy.

    \item Generally, if you're confident at range with your weapon you can static hold angles more effectively, but the real reason is
    peaker's advantage.

    \item If you're static holding an angle and your opponent peaks you, they have the advantage. But if you peek-unpeak an angle, you
    have a chance to peak after your opponent has pushed out, giving you peaker's advantage.
\end{itemize}

\subsection*{Taking Space}
One thing I often see go overlooked is the concept of space. Taking space on a map means controlling where your opponents can and
cannot be at any given time. Leveraging this concept well is fundamental to executing proper CT side holds and, especially, T side
attacks, and is the primary reason many players have trouble transitioning between T and CT sides. 
\begin{itemize}
    \item At the start of a normal bomb plant game Ts are attacking and CTs are defending.
    \item This means that CTs are trying to \textbf{Hold Space} and Ts are trying to \textbf{Take Space}
    \item To this end, many players quickly realize that Smokes and Molotovs are essential to controlling space due to their ability
    to directly deny enemy sight lines and ground space, respectively.
\end{itemize}
    One of the most effective ways to take space in a semi-coordinated setting is to \textbf{Entry Frag} on the location
    you want to take space in. This accomplishes one immediately obvious thing: getting a friendly player in the space you want to
    take. This effectively makes the space a no-mans-land for both teams until one side can confirm there are no enemy players in
    the space anymore. Even if your entry fragger doesn't get the kill, he opens the space up for the rest of his team to push behind
    him during the time he's running out. These few seconds are the most critical to a successful entry frag. Generally, entry fraggers
    work in pairs or with a coordinated team to confirm a trade of players. In most situations the attacking team should have the
    man-advantage when attempting to take an area, since the area that needs to be defended (eg. A and B sites) is much larger than
    the area that needs to be attacked (eg. A \textbf{or} B sites)

\subsection*{Solo Carrying}
There are many skills associated with playing in an uncoordinated setting that are worth examining. People often refer to this
type of playstyle as "solo carrying" due to it's effectiveness at making an impact on rounds to such an extent that the player
who masters these skills appears much more competant than those who focus only on playing in a team. Primary characteristics
of this playstyle include, but are not required or limited to:
\begin{itemize}
    \item utility setups for solo attacking/defending (eg. solo banana as T on inferno)
    \item good risk/reward management / decision making (high-risk high-reward, play making)
    \item good information collection and game awareness (aggressive CT side, counter-strats, quick rotates)
    \item high mechanical skill (aim, flicking, reaction time)
    \item high self-awareness to recongnize when the enemy team is adapting and changing plans
    \item a tilt-proof mindset
\end{itemize}
Probably the most important thing an aspiring solo carry player can aim for is to maximize their solo impact on a game. This means
that they're focusing on value they can guarantee during every game (eg. utility setups) and pushing the game in such a way that
they are able to control the pace of the match (risk-reward management). You can imagine a player who, on T side inferno,
pushes up banana with smokes and mollys, and gets a kill and takes banana control every round. Pratically, if you're succeeding
in getting map control and maybe a pick more than half of the time, you're making a good play and you're probably carrying the game
even if you aren't ace-ing rounds. A really good solo-carry player will be able to snowball a man-advantaged round into a win by
leveraging the other basic CS:GO skills like trading out teammates. This can quickly turn a 4v5 into a 1v2 and sometimes you'll
be able to net extra kills to really lock-down your man-advantage and win rounds handily.

As a final note, the most important thing to do in any game of CS:GO is to remain untilted. Sometimes, your team is throwing harder
than you can carry, and sometimes you just aren't good enough to carry against a well-equipped team. There are risks associated with
playing a solo-carry playstyle and coordinated teams will be able to exploit your high-risk decisions in a way that uncoordinated
teams just aren't equipped to. In these situations it's best to fall back on the CS fundamentals and use your judgement to make
decisions about how best to play against your opponents.

\section*{Teammate Observations}

	This section describes my observations of my teammates decision making during a match. This data is collected
	by re-watching the matches we play as a group and noting each decision the player makes, as well as the outcome
	of that decision. The goal is to establish a clear picture of each player's strengths and weaknesses to both
	identify points that should be worked on for improvement and to understand what situations each member of the
	team is strong in so we can have them in those positions as often as possible.

\subsection{JT}

\subsubsection{Overview}
	Very consistent with scoped weapons. While strong at holding angles with most weapons, JT
	is weak with rifles when compared to his overall preformance. His primary weaknesses appear
	to be rifling at mid and long ranges, and clearing angles while on rotate. He is often killed or forced into
	a scramble situation due to his neglegance when rotating around areas of the map he has little or
	no information about. Despite this weakness he is able to clinch many kills in disadvantaged situations
	due to his superior overall flicking ability and situational awareness
\subsubsection{Strengths}
	Scoped accuracy/flicks, Holding angles, Even duels, HE grenades, map awareness
\subsubsection{Weaknesses}
	Rifle accuracy, Crosshair placement, clearing angles, decisiveness
\subsubsection{Recommended Positons}
	AWPer, late round carry
\subsubsection{Extra Notes}
	JT has a couple of habits that are noteworthy here.

	He likes to take risks by challenging enemy players to duels that he is even or slightly disadvantaged
	in. His willingness to do so is likely influenced by his good flicking ability.

	JT has a tendancy to go for ninja defuses. While not necessarily a bad idea, he should be aware of
	situations where it may be beneficial to attempt to win by taking duels at isolated angles instead.

	Occasionally JT finds himself in a position where he must make a choice on limited information
	In these situations he often does not make a decision quickly, and instead will try to play some
	sort of middle ground between his options while he waits for more info. In these siuations he
	should probably aim to make a decision and rotate quickly, since waiting for more info almost always
	results in him having to play a retake situation.

\subsubsection{Demo Review Notes}

Game 1: Inferno\par
Round 1 - USP, HE Grenade - Loss, Dies\par
Peaks mid, no duel. Walk peak
1 tap headshot while walking up banana
poor crosshair placement when clearing dark, coffin, boostbox
forced aim duel from coffin, loses

R2 - P250, HE - Loss, Dies\par
Lit peaking mid, retreats to arch
close angle on mid, wins duel, +scout
tries to hold at arch, too lit to fight, loses

R3 \$4800 - Helmet, M4A1, HE, Flashx1 - Win, Lives\par
Holding Pit
Headshot on Apts,
Careless peak boiler leads to free shot from enemy in library

R4 \$3850 - Helmet, Ak, P250, Flashx2, Smoke, HE - Loss, Dies\par
Early indecision, last to rotate, Loses rifle duel

R5 \$4950 - M4, HE, Smoke - Loss, Dies\par
Walk peaks duel during rotate, dies

R6 \$2450 - Deagle, HE - Dies, Win\par
Holding angle arch, 1 kill
Re-peak, killed

R7 \$5250 - +Helmet, +Defuser, +M4, +HE, +Smoke - Death, Loss\par
Peak B from CT - Damage 1 enemy
HE on damaged enemy from CT - 
Peak newbox from CT - Damage on 1 enemy, self lit 52
Smoke B while Pushing B from Ct - Doesn't wait for smoke, killed while running onto B

R8 \$1950 - +HE - Lives, Win\par
Lurk arch with teammate - 1 assist, 1 kill +AWP
Peak site from arch - 1 kill

R9 - \$5300 - +Helmet, AWP, HE, Smoke, +Flash - Lives, Win\par
Peak mid - 1 kill
Duel Apts from site - 1 kill

R10 \$7150 - Helmet, Df, AWP, HE, Smoke, Flash, +Flash, +Deag - Lives, Win\par
Peak mid - 1 kill
Watch mid from arch - whiff
scramble at arch side mid - 1 kill

R11 \$9700 - Helmet, Df, AWP, Deag, HE, Smoke, Flashx2 - Dies, Loss\par
Scramble banana - 1 kill - didn't clear the angle logs
Throws all his utility through the smoke and then pushes, dies

R12 \$11400 - +Hemlet +Df +AWP +p250 +HE +Sm +Flx2 - \par

R13\par
Indecisive when enemies spotted at both sites
good crosshair placement

R14\par
Good scope flick accuracy


\subsection{Jonathan}

\subsubsection{Overview}
\subsubsection{Strengths}
\subsubsection{Weaknesses}
\subsubsection{Recommended Positons}
\subsubsection{Extra Notes}
\subsubsection{Demo Review Notes}
\input{}

\include{Jared}
\subsection{Bennett}

\subsubsection{Overview}
His greatest strength is his utility use. Bennett consistently finds ways to use smokes and
molotovs to pressure the enemy team and he isn't afraid to experiment with smokes to find interesting
and useful 1 ways or chokes. That said, he has quite a few issues when fighting toe-to-toe with enemy
players.

In Genereal, he does a couple things wrong when fighting. The first is his ADADAD pattern. He clearly
understands the idea: Strafe and counter-strafe to switch direction and fire a bullet when you're stopped,
but his timing off, so he's usually moving when he fires. The second thing is that he doesn't flick to a target.
(see: aim guide)

I think he should prioritize putting himself in situations where he can push aggressively
with a teammate to ensure a trade.
\subsubsection{Strengths}
Utility, holding angles

\subsubsection{Weaknesses}
aim, awareness

\subsubsection{Recommended Positons}
support, entry fragger

\subsubsection{Extra Notes}
I think in general Bennett plays really passive T side. His weaker aim probably
leads him to avoid duels and engagements that may be necessary to execute agressive T
side attacks. (see: trading)

\subsubsection{Demo Review Notes}
Game 1; Infernoi
Ct Start
R1 \$800 - save - dies, loss
static holding coffin - dies

In the replay here he fires his first bullet before he even moves the mouse. When he does move the mouse it looks
like he's trying to track to the target but he's moving the mouse too slow to catch it as it moves across the screen.
(not flicking, read aim guide)

R2 \$2700 - +p250 - dies, loss
takes duel pushing up mid - dies

R3 \$4800 - +Armor, +DF, +Famas, +Sm, +In - lives, win

R4 \$4050 - Armor DF Famas Sm - dies, loss
Utility B - Holds site
Hides Boostbox - 1 kill, traded

Everything seems fine about this round other than to be aware that there is usually more than 1 person pushing
on a B execute, and to be ready for multiple enemies.

R5 \$5000 - +Helmet, +Df, +Famas, +HE, +Sm, +In - dies, loss
Takes duel Ct arch - loses

R6 \$2750 - +Armor +p250 +Sm +In - dies, wins
pushes banana - +Ak
peeks from arch lurk to mid - killed

R7 \$4400 - +Armor +Df +Famas +HE +Sm +In - dies, loss
Pushes construction - peeks coffin
peeks coffin - 2 kills
walks out of coffin - enemy whiffs shot
pushes dark - dies

R8 \$2600 - +armor +nova +sm - lives, win
pushes mid from banana - 1 kill +awp

R9 \$4750 - Armor, +M4A1-S, +HE, +Sm, +In - lives, win
Takes duel against enemy in fire banana - misses, survives

R10 \$3900 - Armor, +Helmet, +DF, M4, HE, +Sm, +In - lives, win
Takes 2v1 with teammate - 1 kill

R11 \$5150 - Helmet, DF, M4, Deag, HE, +Sm, +In, +Fl - dies, loss
peaks banana - dies

R12 \$5450 - +Armor, +DF, +M4, +HE, +Sm - 1 kill, loss, dies
They plant, Bennett gets the rotate thru Library, - 1 kill
killed by awp from site thru library

R13 \$3100 - +Armor, +DF, +Nova, +Sm - survives, win
1 Kill pushing into mid apts
kills a chicken
smokes banana from ct
teammate gets the kobe for the win

R14 \$4850 - Armor, +Helmet, DF, Ak, +HE, +Sm, +In, +Fl - lives, win
pushes banana
teammate pushes thru smoke and dies
hides at logs until the round is over +awp

R15 \$5700 - Helmet, DF, +M4, HE, Sm, In, Fl - 2 kills, win
sits newbox,
Enemy awp pushes to site, they fight, 1 kill
another guy pushes, 1 kill
They catch the last guy in ct on the rotate

Side Switch - T

R16 \$800 - save - 1 kill, win, dies
runs up mid with his whole team thru boiler onto site,
1 kill on the guy lurking site
killed in cemtary by the guy apts

R17 \$4350 - +Helmet, +Ak, +Mv, +Fl - loss, dies
pushes alt mid killed by ct in apts

R18 \$3300 - +Helmet, +Mac10, +Sm, +Mv, +Fl - win
plants the bomb and never sees an enemy +M4A1

R19 \$3900 - Helmet, M4A1, +Sm, +Mv, +Flx2 - win
walks around the map with two teammates
while they one tap everyone on the enemy team
plants the bomb

R20 \$7450 - Helmet, M4, Sm, Mv, Flx2 - 



\include{Carson}

\end{document}
